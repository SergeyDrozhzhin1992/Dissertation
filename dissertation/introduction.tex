\chapter*{Введение}                         % Заголовок
\addcontentsline{toc}{chapter}{Введение}    % Добавляем его в оглавление

\newcommand{\actuality}{}
\newcommand{\progress}{}
\newcommand{\aim}{{\textbf\aimTXT}}
\newcommand{\tasks}{\textbf{\tasksTXT}}
\newcommand{\novelty}{\textbf{\noveltyTXT}}
\newcommand{\influence}{\textbf{\influenceTXT}}
\newcommand{\theoryInfluence}{\textbf{\theoryInfluenceTXT}}
\newcommand{\methods}{\textbf{\methodsTXT}}
\newcommand{\defpositions}{\textbf{\defpositionsTXT}}
\newcommand{\reliability}{\textbf{\reliabilityTXT}}
\newcommand{\probation}{\textbf{\probationTXT}}
\newcommand{\contribution}{\textbf{\contributionTXT}}
\newcommand{\publications}{\textbf{\publicationsTXT}}
\newcommand{\volumeAndStructureWork}{\textbf{\volumeAndStructureWorkTXT}}


Слово репликация происходит от латинского \textit{Replicatio}, что переводится как возобновление или повторение. Данный термин применяется в различных областях. Например, в вычислительной технике --- это механизм синхронизации содержимого нескольких копий объекта, в изобразительном искусстве --- один из способов образования художественной формы, берущий в качестве образца ранее существовавшее или существующее произведение. Нас будет интересовать репликация как термин биологической эволюции, который применяется к процессу удвоения молекулы ДНК. Данный термин определяет понятие ``репликатор''\,, которое в разных источниках определяется по - разному. Так в работе (\cite{Markov}, стр. 656) репликатор определяется как объект, обладающий способностью самовоспроизводится и наследственной устойчивостью. А, например, в работе (\cite{Deutsch}, стр. 390) --- ``любой объект, побуждающий особые среды его копировать''\,. 

Для вывода уравнения репликаторной системы в общем виде, воспользуемся уравнением Т.~Мальтуса

\begin{equation}
\frac{dN_{i}}{dt} = g_{i}(\mathbf {N})N_{i}, \quad \mathbf {N} = \Big(N_{1}(t), ..., N_{n}(t)\Big),
\label{intr_eq1}
\end{equation}
где $N_{i}(t), i = \overline{1, n}$ --- численность популяции $i$-го вида, а $g_{i}(\mathbf {N})$ --- непрерывные функции, описывающие взаимодействие видов $g_{i}: \mathbb{R}_{+}^{n} \to \mathbb{R}$. Кроме того, будем считать, что для функций $g_{i}(\mathbf {N})$ выполняется условие однородности порядка $k$, то есть $g_{i}(\xi \mathbf {N}) = \xi^{k}g_{i}(\mathbf {N}), \xi \in \mathbb{R}$. 

Обозначим за $u_{i}(t)$ относительную частоту $i$-го вида

\begin{equation}
u_{i}(t) = \frac{N_{i}(t)}{\sum\limits_{j = 1}^{n}N_{j}(t)},
\label{intr_eq2}
\end{equation} 
тогда, учитывая условие однородности функций $g_{i}(\mathbf {N})$, систему \eqref{intr_eq1} можно записать в виде

\begin{equation}
\frac{du_{i}}{dt} = \Bigg(\sum\limits_{j = 1}^{n}N_{j}(t)\Bigg)^{k} \Bigg(g_{i}(\mathbf {u})u_{i} - u_{i}\sum\limits_{j = 1}^{n}g_{j}(\mathbf {u})u_{j}\Bigg), \quad i = \overline{1, n}.
\label{intr_eq3}
\end{equation} 
В силу того, что $\sum\limits_{j = 1}^{n}N_{j}(t) > 0$, полученная система \eqref{intr_eq3} будет орбитально топологически эквивалентна \cite{Arnold} системе

\begin{equation}
\frac{du_{i}}{dt} = u_{i}\Big(g_{i}(\mathbf {u}) - f(t)\Big), \quad f(t) = \sum\limits_{j = 1}^{n}g_{j}(\mathbf {u})u_{j},
\label{intr_eq4}
\end{equation}

$$
\sum\limits_{j = 1}^{n}u_{j}(t) = 1, \quad u_{i}(0) = u_{i}^{0}, \quad i = \overline{1, n}.
$$ 
Свойство эквивалентности систем \eqref{intr_eq3} и \eqref{intr_eq4} заключается в гомеоморфизме их фазовых пространств с сохранением положений равновесия и их характера.

Положив в формуле \eqref{intr_eq4} $g_{i}(\mathbf {u}) = \Big(\mathbf {Au}\Big)_{i} = \sum\limits_{j = 1}^{n}a_{ij}u_{j}$ получим общий вид репликаторной системы

\begin{equation}
\frac{du_{i}}{dt} = u_{i}\Bigg[\Big(\mathbf {Au}\Big)_{i} - f(\mathbf {u})\Bigg], \quad u_{i}(0) = u_{i}^{0}, \quad i = \overline{1, n},
\label{intr_eq5}
\end{equation}
где $\mathbf {A} = ||a_{ij}||_{i, j = 1,...,n}$. Из формулы \eqref{intr_eq2} следует, что решения системы \eqref{intr_eq5} разыскиваются на симплексе $S_{n} = \Big\{u_{i}(t) \ge 0, i = \overline{1, n}, \sum\limits_{i = 1}^{n}u_{i}(t) = 1\Big\}$.

Величина $f(\mathbf {u})$ --- это \textit{средняя приспособленность (фитнес)} системы, которая определяется как
$$
\quad f(\mathbf {u}) = \Big(\mathbf {Au, u}\Big).
$$
В свою очередь функция $\Big(\mathbf {Au}\Big)_{i}$ называется \textit{приспособленностью (фитнесом)} $i$-го вида и определяется матрицей взаимодействия $\mathbf {A}$, которая также называется \textit{ландшафтом приспособленности}. Элемент $a_{ij}$ матрицы $\mathbf {A}$ задает влияние $j$-го вида на вид с номером $i$.

Частным случаем репликаторной системы является \textit{гиперциклическая} репликаторная система, которая занимает центральное место в данном исследовании. \textit{Гиперцикл} --- это система, в которой каждая макромолекула катализируется с помощью предыдущей и все это происходит в замкнутом цикле. Данные системы являются примером альтруистического типа поведения. Позднее будет показано, что гиперциклические системы удовлетворяют триаде Ч.~Дарвина: \textit{наследственность, изменчивость и естественный отбор}.

Первыми, кто изучал такие системы, были М.~Эйген и П.~Шустер (\cite{Eig4}, \cite{Eig5}, \cite{Shuster2}) --- основоположники предбиологической теории эволюции.

{\aim} данной работы является исследование процесса эволюции ландшафта приспособленности перманентной (невырожденной) репликаторной системы, направленного на увеличение средней приспособленности (фитнеса) системы.


Для~достижения поставленной цели необходимо было решить следующие {\tasks}:
\begin{enumerate}
  \item Изучить существующие работы по математическому моделированию эволюции
репликаторных систем.
  \item Разработать математическую модель эволюции репликаторной системы, основанную на возможности адаптивного изменения параметров системы.
  \item Разработать численный метод решения соответствующих систем уравнений.
  \item Программно реализовать численные методы моделирования эволюции репликаторных систем.
  \item Проанализировать и проинтерпретировать полученные результаты.
\end{enumerate}


{\novelty}
\begin{enumerate}
  \item Впервые для исследования процесса эволюции ландшафта приспособленности репликаторной системы применяется метод, основанный на возможности адаптивного изменения параметров системы. Для решения поставленной задачи вводится гипотеза о существовании специального времени эволюционной адаптации параметров системы, которое гораздо более медленное, чем время описывающее внутреннюю динамику системы. Это время мы будем называть эволюционным временем. Данное предположение приводит к тому факту, что эволюционные изменения параметров системы происходят в стационарном положении равновесия рассматриваемой динамической системы. Таким образом, исходная постановка задачи о максимизации средней приспособленности может быть интерпретирована как задача о максимизации средней приспособленности в стационарном положении равновесия путем варьирования параметров системы в условиях ограниченности значений этих параметров. Данная задача сводится к серии задач линейного программирования, для решения которых применяются методы численного моделирования и разработана компьютерная программа.
  
  \item Разработан численный метод для описания процесса эволюция гиперциклической репликаторной системы в условиях присоединения новых видов в случайные моменты времени. Кроме того, описаны условия, при которых система остается перманентной в результате добавления нового элемента.  
   
  \item Исследован процесс эволюции бигиперциклической репликаторной системы. Разработана соответствующая математическая модель и описан процесс численного метода решения соответствующих уравнений.
\end{enumerate}


{\theoryInfluence}
Полученные результаты вносят вклад в развитие, так называемой, теории предбиологической эволюции, предложенной М.~Эйгеном. Эти результаты доказывают, что в результате предложенного эволюционного процесса репликаторная система становится резистентной к воздействию паразитических макромолекул. Ранее факт нуестойчивости репликаторных систем к воздействию макромолекул - паразитов представлял основное препятствие в развитии этой теории.


{\influence}
Полученные результаты могут быть применены к проблеме эволюции болезнетворных бактерий и клеток под действием лекарственных средств.


{\defpositions}
\begin{enumerate}
\item Разработана математическая модель эволюции гиперциклической репликаторной системы. Рассмотрен вопрос, связанный с необходимым и достаточным условием достижения экстремума функции средней приспособленности. Представлены многочисленные примеры эволюционного изменения системы. 

\item Разработана математическая модель эволюции бигиперциклической репликаторной системы. Рассмотрен вопрос, связанный с необходимым и достаточным условием достижения экстремума функции средней приспособленности. Представлены многочисленные примеры эволюционного изменения системы. 

\item Разработана математическая модель эволюции гиперциклической репликаторной системы, в условиях появления новых видов в случайные моменты времени. Исследованы условия, при которых система остается перманентной после добавления нового вида. Представлены многочисленные примеры эволюционного изменения системы. 

\item Исследованы случаи двух репликаторных систем, первая из которых напоминает структуру биологического сообщества, управление которым происходит с помощью одного, центрального элемента --- ``муравейник''\,, а вторая система взята из результатов конкретной системы, полученной биохимическим путем и задающая репликацию реальных объектов \cite{Vaidya}. 
\end{enumerate}


{\probation}
Основные результаты работы докладывались~на:
\begin{itemize}

\item  научной конференции ``Тихоновские чтения 2017''\, (Москва, МГУ им. М.~В.~Ломоносова, 23 - 27 октября 2017 г.)

\item  научной конференции ``Ломоносовские чтения 2018''\, (Москва, МГУ им. М.~В.~Ломоносова, 16 - 27 апреля 2018 г)

\item  международной научной конференции ``“Математическая биология и биоинформатика''\, (Москва, Пущино, 2018 г.)

\item International Scientific Conference ``Evolving life: the evolution with trade-offs, frustration in selection and growing complexity''\, (Yerevan, Armenia, March 29 - April 3, 2019)

\item  научной конференции ``Ломоносовские чтения 2019''\, (Москва, МГУ им. М.~В.~Ломоносова, 15 - 25 апреля 2019 г.)

\item  международной научной конференции, посвященной 80---летию академика В.~А.~Садовничего ``Современные проблемы математики и механики''\, (Москва, МГУ им. М.~В.~Ломоносова, 13 - 15 мая 2019 г.)

\item  международной научной конференции ``Математическое моделирование в биомедицине''\, (Москва, РУДН, 30 сентября - 4 октября 2019 г.)

\item 19th International Symposium on Mathematical and Computational Biology: BIOMAT 2019\, (Szeged, Hungary, 21 - 25 october 2019)

\end{itemize}

{\publications} \ldots

{\contribution}
Личный вклад автора состоит в разработке математических моделей эволюции репликаторных систем, представленных в главах \ldots, разработке численного метода решения соответствующих систем уравнений и его программной реализацией. Постановка и ход научных исследований осуществлялись под руководством д.ф.- м.н, проф. А.~С.~Братуся. Все основные результаты этих глав опубликованы в статьях \ldots в соавторстве с д.ф.- м.н, проф. А.~С.~Братусем и к.ф.- м.н Т.~Якушкиной, участвовавших в обсуждении результатов и вносивших ценные замечания. 

{\volumeAndStructureWork}
Диссертация состоит из \ldots





