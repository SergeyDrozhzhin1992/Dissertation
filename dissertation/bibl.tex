\begin{thebibliography}{99}\itemsep=-2pt

\bibitem{ao1} Ao P.\:, Laws in Darwinian evolutionary theory ~// Physics of life Reviews. 2005. N~2(2). P.\, 117--156.

\bibitem{ao2} Ao P.\:, Global view of bionetwork dynamics: adaptive landscape ~// Journal of Genetics and Genomics. 2009. N~36(2). P.\, 63--73.

\bibitem{Arnold} Арнольд В.\:, Обыкновенные дифференциальные уравнения. Главная редакция физико-математической литературы издательства ``Наука''\,, 1971.

\bibitem{Baake} Baake E., Wagner H.\:,  Mutation–selection models solved exactly with methods of statistical mechanics ~// Genet. Res. 2001. N~78(1). P.\,93--117.

\bibitem{Boerlijst} Boerlijst M.~C., Hogeweg P.\:, Spiral wave structure in pre-biotic evolution: Hypercycles stable against parasites ~// Phys. Nonlinear Phenom. 1991. N~48. P.\, 17--28.

\bibitem{BratusA} Bratus A.~S., Semenov Y.~S., Novozhilov A.~S.\:, Adaptive fitness landscape for replicator systems: to maximize or not to maximize ~// Mathematical Modelling of Natural Phenomena. 2018. N~13(3). P.\,25.

\bibitem{Bratus1} Bratus A.~S., Novozhilov A.~S., Semenov Y.~S.\:, Linear algebra of the permutation invariant crow–kimura model of prebiotic evolution ~// Mathematical Biosciences. 2014. N~256. P.\,42--57.

\bibitem{Bratus2} Bratus A.~S., Novozhilov A.~S., Semenov Y.~S.\:, On the behavior of the leading eigenvalue of Eigen’s evolutionary matrices ~// Mathematical Biosciences. 2014. N~258. P.\,134--147.

\bibitem{Bratus3} Bratus A.~S., Posvyanskii V.~P.\:, Stationary solutions in a closed distributed Eigen - Shuster evolution systems ~// Differential Equations. 2006. N~42. P.\,1762--1774.

\bibitem{Bratus4} Bratus A.~S., Posvyanskii V.~P., Novozhilov A.~S.\:, Existence and stability of stationary solutions to spatially extended autocatalytic and hypercyclic systems under global regulation and with nonlinear growth rates ~// Nonlinear Analysis: Real World Applications. 2010. N~11. P.\,1897--1917.

\bibitem{Bresch} Bresch C., Niesert U., Harnasch D.\:, Hypercycles, parasites and packages ~// J. Theor. Biol. 1980. N~85. P.\, 399--405.

\bibitem{Brich} Brich J.\: Natural selection and maximization of fitness ~// Biological Reviews. 2016. N~91(3). P.\,712--727.

\bibitem{Cressman} Cressman R.\:, Evolutionary Dynamics and Extensive Form Games. MIT Press Series on Economic Learning and Social Evolution, 5, MIT Press, Cambridge, 2003.

\bibitem{Crow} Crow J. , Kimura M.\: An Introduction to Population Genetics Theory. Harper and Row, Publishers, New York, Evanston and London, 1970.

\bibitem{Darwin} Darwin C.\: On the Origin of Species by Means of Natural Selection, or the Preservation of Favoured Races in the Struggle for Life. London: John Murray, 1859.

\bibitem{Deutsch} Deutsch D.\: The fabric of reality. New-York: Allen Lane, 1997.

\bibitem{Eig1} Eigen M.\: Self-organization of matter and the evolution of biological macro-molecules~// Naturwissenschaften. 1971. N~58. P.\,465--532.

\bibitem{Eig2} Eigen M., McCascill J., Schuster P.\:, Molecular quasi-species~// J. Phys. Chem. 1988. N~24. P.\, 6881--6891.

\bibitem{Eig3} Eigen M., McCascill J., Schuster P.\:, The molecular quasi-species~// Adv. Chem. Phys. 1989. N~75. P.\, 149--263.

\bibitem{Eig4} Eigen M., Schuster P.\: The Hypercycle: A principal of natural self-organization. Part A: emergence of the hypercycle~// Naturwissenschaften. 1978. N~64. P.\, 541--565. 

\bibitem{Eig5} Eigen M., Schuster P.\: The Hypercycle: A principal of natural self-organization. Part B: the abstract hypercycle~// Naturwissenschaften. 1978. N~65. P.\, 7--41. 

\bibitem{ewens} Ewens W.~J.\:, An interpretation and proof of the fundamental theorem of natural selection ~// Theoretical Population Biology. 1989. N~36(2). P.\, 167--180.

\bibitem{Fisher} Fisher R.\: The Genetical Theory of Natural Selection. Oxford: At The Clarendon Press, 1930.

\bibitem{Galluccio} Galluccio S.\:, Exact solution of the quasispecies model in a sharply peaked fitness landscape ~// Phys. Rev. 1997. N~56(4). P.\,4526.

\bibitem{Garcia} Garcia-Pelayo R., S.\: A linear algebra model for quasispecies ~// Physica A: Statistical Mechanics and its Applications. 2002. N~309(1). P.\,131--156.

\bibitem{grafen1} Grafen A.\:, Optimization of inclusive fitness ~// Journal of Theoretical Biology. 2006. N~238(3). P.\, 541--563. 

\bibitem{grafen2} Grafen A.\:, The simplest formal argument for fitness optimization ~// Journal of genetics. 2008. N~87(4). P.\, 421--433.

\bibitem{Hofbauer1} Hofbauer J., Sigmund K.\: Evolutionary Games and Population Dynamics. Cambridge University Press, Cambridge, 1988.

\bibitem{Hofbauer2} Hofbauer J., Sigmund K.\: Evolutionary game dynamics~// Bulletin of American Mathematical Society. 2003. N~40. P.\, 479--519.

\bibitem{Hogeweg} Hogeweg P., Takeuchi N.\:, Multilevel selection in models of prebiotic evolution: compartments and spatial self-organization ~// Orig. Life Evol. Biosphere. 2003. N~33. P.\, 375--403.

\bibitem{Lamark} Lamarck J. ~B.\: Philosophie zoologique. Dentu, 1809.

\bibitem{lessard} Lessard S.\:, Fisher's fundamental theorem of natural selection revisted ~// Theoretical Population Biology. 1997. N~52(2). P.\, 119--136.

\bibitem{Leuth1} Leuthausser I.\:,  An exact correspondence between Eigen's evolution model and a two---dimensional Ising system ~// J. Chem. Phys. 1986. N~84(3). P.\,1884--1885.

\bibitem{Leuth2} Leuthausser I.\:,  Statistical mechanics of Eigen’s evolution model ~// J. Stat. Phys. 1987. N~48(1). P.\,343--360.

\bibitem{Lincoln} Lincoln A., Joyce G.\:, Self-Sustained Replication of an RNA Enzyme ~// Science. 2009. N~323. P.\, 1229--1232.

\bibitem{Malthus} Malthus T.\: An Essay on the Principle of Population, as it affects the future Improvement of Society. London: J.~Johnson, 1798.

\bibitem{Markov} Марков А.~Н., Неймарк Е.\: Эволюция: Классические идеи в свете новых открытий. АСТ: Corpus, 2014.

\bibitem{Maynard} Maynard J.~S.\:, Evolution and the Theory of Games. Cambridge University Press, 1982. 

\bibitem{Parker} Parker G.~A., Smith J.~M.\:, Optimality theory in evolutionary biology ~// Nature. 1990. N~348(6296). P.\, 27--33.

\bibitem{Rumschitzki} Rumschitzki D.\: Spectral properties of Eigen evolution matrices ~// Journal of Mathematical Biology. 1987. N~24(6). P.\,667--680.

\bibitem{Shuster1} Swetina J., Schuster P.\:,  Self-replication with errors: A model for polvnucleotide replication ~// Biophysical Chemistry. 1982. N~16(4). P.\,329--345.

\bibitem{Shuster2} Shuster P.\: Mathematical modeling of evolution. Solved and open problems~// Theory Biosci. 2011. N~130. P.\, 71--89. 

\bibitem{Shuster3} Shuster P., Sigmund K.\: Replicator dynamics~// Journal of Theoretical Biology. 1983. N~100. P.\, 533--538.

\bibitem{Sigmund} Sigmund K.\:, The calculus of selfishness. Princeton University Press, Princeton, 2010.

\bibitem{Svirezhev} Svirezhev Y.~M., Passekov V.~P.\: Fundamentals of Mathematical Evolutionary Genetics. Kluwer Academic Publishers, Dordrecht, 1990.

\bibitem{Szab} Szab{\'o} P., Scheuring I., Cz{\'a}r{\'a}n T., Szathm{\'a}ry E.\:, In silico simulations reveal that replicators with limited dispersal evolve towards higher efficiency and fidelity ~// Nature. 2002. N~420. P.\, 340--343. 

\bibitem{Takeuchi} Takeuchi N., Hogeweg P.\:, Multilevel Selection in Models of Prebiotic Evolution II: A Direct Comparison of Compartmentalization and Spatial Self-Organization~// PLoS Comput Biol. 2009. N~5(10).e1000542.

\bibitem{Taylor} Taylor P., Jonker L.\:, Evolutionarily stable strategies and game dynamics ~// Mathematical Biosciences. 1978. N~40. P.\, 145--156.

\bibitem{Vaidya} Vaidya N., Manapat M., Chen. I., Xulvi-Brunet R., Hayden E., Lehman N.\:, Spontaneous network formation among cooperative RNA replicators ~// Nature. 2012. N~491. P.\, 72--77.

\bibitem{Verhlust} Verhlust P.\: Notice sur la loi que la population poursuit dans son accroissement~// Corresp. Math. Phys. 1838. N~10. P.\, 113-121.

\bibitem{wright} Wright S.\:, The roles of mutation, inbreeding, crossbreeding and selection in evolution ~// Proc. Sixth Int. Congr. Gen. 1932. P.\, 356--366.

\end{thebibliography}